\documentclass[a4paper,spanish] {article} 
\usepackage [spanish] {babel} 
\usepackage [latin1]{inputenc}
\usepackage{graphicx}
\usepackage{caratula}
\usepackage{subfig}
\usepackage{dsfont}
\usepackage{algorithm}
\usepackage{amsmath}
\usepackage{algorithmic}

\addtolength{\oddsidemargin}{-1in}
\addtolength{\textwidth}{2in}

\begin{document}
\pagestyle{headings}



\newpage

\materia{Algoritmos y Estructuras de Datos III}
\submateria{Primer Cuatrimestre del 2010}
\titulo{Trabajo Pr�ctico 3}

\integrante{Nicol�s Arias}{369/08}{nik\_261@hotmail.com}
\integrante{Ezequiel Castellano}{161/08}{ezequielcastellano@gmail.com}
\integrante{Carolina Hadad}{367/08}{carolinahadad@gmail.com}
\integrante{Nicol�s Papagna Maldonado}{424/03}{nicolas.papagna@gmail.com}

\maketitle


\newpage
\tableofcontents
\newpage

\section{Introducci�n al problema de MAX-CLIQUE}
	En teor�a de grafos, un clique en un grafo no dirigido G es un conjunto de v�rtices V tal que para todo par de v�rtices de V, existe una arista que las conecta. En otras palabras, un clique es un subgrafo en que cada v�rtice est� conectado a cada otro v�rtice del grafo. Esto equivale a decir que el subgrafo inducido por V es un grafo completo. El tama�o de un clique es el n�mero de v�rtices que contiene.En este trabajo practico nos interesa buscar el clique m�ximo de un grafo: su m�ximo subgrafo completo.
	
	Veamos un ejemplo:
	\begin{figure}[h!]
\centering
    \includegraphics{cliquesample.PNG}
\caption{Ejemplo clique m�ximo}
\end{figure}

	En este caso tenemos un grafo con 10 nodos cuyo m�ximo clique tiene 4 nodos. Se puede ver que no existe un subconjunto de m�s nodos que cumpla que cada uno de ellos est� relacionado con todos los otros.
	
	Pese a tener muchas aplicaciones en ciencia e ingenier�a (algunas de las cuales detallaremos en la secci�n siguiente), este problema no est� computacionalmente bien resuelto todav�a. Aunque existen algoritmos m�s eficientes que los de fuerza bruta para resolverlo de manera exacta, todos ellos toman un tiempo exponencial en resolver el problema. Por esta raz�n, se desarrollan muchos algoritmos para resolver el problema eficientemente para grafos particulares. En este trabajo pr�ctico desarrollaremos algoritmos exactos y heur�sticos para solucionarlo.
	
\section{Ejercicio 1: Aplicaciones del problema}

	\subsection{Aplicacion de nico A}
	
	\subsection{Aplicacion de caste}

	\subsection{Aplicacion de nico P}

	\subsection{Aplicacion de caro}



	 
	
\end{document}