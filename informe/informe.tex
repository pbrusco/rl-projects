\documentclass[a4paper,spanish] {article} 
\usepackage [spanish] {babel} 
\usepackage [latin1]{inputenc}
\usepackage{graphicx}
\usepackage{caratula}
\usepackage{subfig}
\usepackage{dsfont}
\usepackage{algorithm}
\usepackage{amsmath}
\usepackage{algorithmic}

\addtolength{\oddsidemargin}{-1in}
\addtolength{\textwidth}{2in}

\begin{document}
\pagestyle{headings}



\newpage

\materia{Aprendizaje por Refuerzos: Teor�a y Aplicaciones en Rob�tica, Psicolog�a y Neurociencias}
\submateria{Tp Final}
\titulo{Desarrollo de algoritmos de aprendizaje y an�lisis de resultados en una adaptaci�n del problema Bomberman}


\integrante{Carolina Hadad}{367/08}{carolinahadad@gmail.com}

\maketitle

\section{Introducci�n}
	En  este Trabajo Practico nos interesa analizar el desempe�o de los algoritmos de aprendizaje aprendidos durante el curso,  aplic�ndolos  en un problema distinto distinto de los vistos; en nuestro caso, elegimos adaptar el juego del Bomberman. 
		\subsection{Adaptaci�n del juego del Bomberman}
		Para probar los algoritmos implementados, buscamos un problema distinto a los vistos en el curso pero con caracter�sticas similares, para poder comparar nuestros resultados con los conocidos?.
				
		El juego que vamos usar es una versi�n simplificada del juego del Bomberman. El objetivo del agente es llegar a la salida. El agente tiene 4 acciones de movimiento: arriba, abajo, hacia la izquierda y hacia la derecha. El tablero tiene paredes que obstaculizan su camino, algunas de ellas son rompibles y otras irrompibles. El agente tiene acciones para poner una bomba y para explotarla. Solo puede haber una bomba en el tablero, si el agente realiza una acci�n de tirar bomba habiendo una bomba en el juego, la acci�n no tendr� efecto. Al explotar la bomba se destruir�n las paredes que est�n arriba, abajo a la izquierda y a la derecha de ella. Si el agente estuviera en alguno de estos lugares o sobre la bomba, el agente muere, teniendo que empezar el juego nuevamente desde la posici�n inicial. Nos interesa que el agente aprenda una pol�tica �ptima para llegar a la salida realizando la menor cantidad de acciones posibles.
		
		
		

		
		
		
		
\newpage
\tableofcontents
\newpage
	 
	
\end{document}