\subsubsection{RMax}

Para implementar Rmax en nuestro trabajo, decidimos basarnos en la versi�n de KWIK-Rmax presente en las slides de la quinta clase del curso. Adem�s, para m�s detalle sobre KWIK, y sobre las cotas matem�ticas necesarias, nos basamos fuertemente en una disertaci�n de Lihong Li \cite{li}.

Un agente Rmax, al ser model based, va armando un modelo completo en base a la experiencia adquirida y, al momento de decidir qu� acci�n tomar, elige aqu�lla que considera m�s valiosa en base a todo el modelo que tiene computado el agente. Entonces, el algoritmo se divide en dos fases: una fase de aprendizaje, donde actualizamos el modelo en base a la experiencia emp�rica, y la fase de toma de decisi�n, donde, en base al modelo, mediante alguna t�cnica como value iteration, decidimos qu� acci�n es la que m�s conviene tomar.

Al usar una versi�n KWIK para Rmax con las cotas necesarias, se garantizar�a que el algoritmo completo fuera PAC. 

\subsubsection{RMax Factorizado}